\documentclass[a4paper,usenames,dvipsnames,11pt]{article}
\pdfoutput=1

%% packages
\usepackage{jheppub}
\usepackage{slashed}
\usepackage{mathrsfs,booktabs,multirow,tabularx}
\usepackage{stmaryrd}
\usepackage{xspace}
\usepackage{fancyvrb}
\usepackage[makeroom]{cancel}
\usepackage{amsmath}    % need for subequations
\usepackage{amssymb}    % 
\usepackage{graphicx}   % need for figures
\usepackage{verbatim}   % useful for program listings
\usepackage{lscape}
\usepackage{subfig}
\usepackage{listings}
\usepackage{mathtools}
\usepackage{enumerate}
\usepackage{draftwatermark}

%\usepackage[nomarkers]{endfloat}


%
\renewcommand\arraystretch{1.1}
%\setlength{\textfloatsep}{10pt plus 1.0pt minus 2.0pt}

%NOTE: \beq and \eeq will not work if using amsmath. 
%Use \beqn and \eeqn instead
\def\beq{\begin{equation}}
\def\eeq{\end{equation}}
\def\beqn{\begin{eqnarray}}
\def\eeqn{\end{eqnarray}}
\def\abs#1{\left|#1\right|}
\def\IN{{\small IN}}
\def\OUT{{\small OUT}}
\def\remove#1#2{#1\hspace{-#2truecm}\backslash}
\def\removeb#1#2#3{#1\hspace{-#2truecm}\backslash\hspace{-#3truecm}\slash}
\def\red{}
\def\xidistr#1{\left(\frac{1}{\xi}\right)_{\!#1}}
\def\pdistr#1#2{\left(\frac{1}{#1}\right)_{\!#2}}
\def\lxidistr#1{\left(\frac{\log\xi}{\xi}\right)_{\!#1}}
\def\lpdistr#1#2{\left(\frac{\log#1}{#1}\right)_{\!#2}}
\def\lppdistr#1#2{\left(\frac{\log\left(#1\right)}{#1}\right)_{\!#2}}
\def\ypdistr#1{\left(\frac{1}{1-y}\right)_{\!#1}}
\def\FKSeq#1{eq.~({\bf II}.#1)}
\def\MadFKSeq#1{eq.~({\bf I}.#1)}
\def\set#1#2#3{\left\{#1\right\}_{#2,#3}}
\def\setS#1#2#3#4{\left\{#1\right\}_{#2,#3}^{[#4]}}
\def\oset#1#2#3{\left(#1\right)_{#2,#3}}
\def\binomial#1#2{
\left(\!\!
\begin{array}{c}
#1\\
#2
\end{array}
\!\!\right)
}

%%CMS stuff (to be removed or included elsewhere)
\newcommand{\bqa}{\begin{eqnarray}}
\newcommand{\eqa}{\end{eqnarray}}
% A small hack to have coloured comment in the tex file  without using colour package
% It's ok since there will be no comments left for the final version
\chardef\MyArticleWithColor=\pdfcolorstackinit page direct{0 g}
\def\cmtVH#1{\emph{\pdfcolorstack\MyArticleWithColor push {1 0 0 rg} V.H. : #1 \pdfcolorstack\MyArticleWithColor pop}}
\def\cmtHS#1{\emph{\pdfcolorstack\MyArticleWithColor push {0 1 0 rg} H.S. : #1 \pdfcolorstack\MyArticleWithColor pop}}
\newcommand{\plaat}[3]{\raisebox{#3pt}{\epsfig{figure=#1.pdf,width=#2cm}}}
%%\newcommand{\plaat}[3]{\raisebox{#3pt}{\epsfig{figure=#1.eps,width=#2cm}}}
%%\newcommand{\plaat}[3]{\raisebox{#3pt}{\includegraphics[width=#2cm]{#1-eps-converted-to.pdf}}}
\def\cCode#1{\begin{lstlisting}[mathescape,basicstyle=\small
\ttfamily,frame=leftline,aboveskip=4mm,belowskip=4mm,xleftmargin=20pt,framexleftmargin=10pt,
numbers=none,framerule=2pt,abovecaptionskip=0.0mm,belowcaptionskip=3.5mm #1]}


\newcommand\sss{\scriptscriptstyle}
%\newcommand\sss{}
\newcommand\mydot{\!\cdot\!}
\newcommand\ep{\epsilon}
\newcommand\half{\frac{1}{2}}
\newcommand\quarter{\frac{1}{4}}
\newcommand\as{\alpha_{\sss S}}
\newcommand\gs{g_{\sss S}}
\newcommand\aW{\alpha_{\sss W}}
\newcommand\gW{g_{\sss W}}
\newcommand\aem{\alpha}
%\newcommand\alo{\alpha_1}
%\newcommand\alt{\alpha_2}
\newcommand\alo{\as}
\newcommand\alt{\aem}
\newcommand{\tev}{\,\textrm{TeV}}
\newcommand{\gev}{\,\textrm{GeV}}
\newcommand{\mev}{\,\textrm{MeV}}
\newcommand\bQ{\bar{Q}}
\newcommand\bt{\bar{t}}
\newcommand\aNLO{{\sc\small MadGraph5\_aMC@NLO}}
\newcommand\aNLOs{{\sc\small MG5\_aMC}}
\newcommand\MadGraph{{\sc\small MadGraph}}
\newcommand{\mgatnlo}{\aNLOs\xspace}
\newcommand{\mg}{\aNLOs\xspace}
\newcommand\UFO{{\sc\small UFO}}
\newcommand\mf{{\sc\small MadFKS}}
\newcommand\ml{{\sc\small MadLoop}}
\newcommand\ct{{\sc\small CutTools}}
\newcommand\nin{{\sc\small Ninja}}
\newcommand\collier{{\sc\small Collier}}
\newcommand\IREGI{{\sc\small IREGI}}
\newcommand\aMCSusHi{{\sc\small aMCSusHi}}
\newcommand\SusHi{{\sc\small SusHi}}
\newcommand\mgamc{\aNLO}
\newcommand\HWpp{{\sc\small Herwig++}}
\newcommand\HWsv{{\sc\small Herwig7}}
\newcommand\PYe{{\sc\small Pythia8}}
\newcommand\HWs{{\sc\small Herwig6}}
\newcommand\FJ{{\sc\small FastJet}}
\newcommand\lhapdfs{{\sc\small LHAPDF6}}
\newcommand\recola{{\sc\small RECOLA}}
\newcommand\sherpa{{\sc\small Sherpa}}
\newcommand\alpgen{{\sc\small AlpGen}}
\newcommand\gosam{{\sc\small GoSam}}
\newcommand\maddip{{\sc\small MadDipole}}
\newcommand\OL{{\sc\small OpenLoops}}
\newcommand{\FeynRules}{{\sc \small FeynRules}}
\newcommand{\feynrules}{\FeynRules}
\newcommand{\nloct}{{\sc \small NloCt}}
\newcommand\prompt{{\tt MG5\_aMC>}}
\newcommand\pt{p_{\sss T}}
\newcommand\pti{p_{{\sss T},i}}
\newcommand\kt{k_{\sss T}}
\newcommand{\Ht}{H_{\sss T}}
\newcommand{\LO}{{\rm LO}}
\newcommand{\LOi}{{\rm LO}_i}
\newcommand{\LOipo}{{\rm LO}_{i+1}}
\newcommand{\LOo}{{\rm LO}_1}
\newcommand{\LOt}{{\rm LO}_2}
\newcommand{\LOth}{{\rm LO}_3}
\newcommand{\LOf}{{\rm LO}_4}
\newcommand{\NLO}{{\rm NLO}}
\newcommand{\NLOi}{{\rm NLO}_i}
\newcommand{\NpLOi}{{\rm N}^p{\rm LO}_i}
\newcommand{\NLOipo}{{\rm NLO}_{i+1}}
\newcommand{\NLOo}{{\rm NLO}_1}
\newcommand{\NLOt}{{\rm NLO}_2}
\newcommand{\NLOth}{{\rm NLO}_3}
\newcommand{\NLOf}{{\rm NLO}_4}
\newcommand{\NLOfv}{{\rm NLO}_5}
\newcommand{\NLOgt}{{\rm NLO}_{\ge 2}}
\newcommand{\NLOgth}{{\rm NLO}_{\ge 3}}
\newcommand{\MSb}{\overline{\rm MS}}
\newcommand{\epUV}{\varepsilon_{\rm\sss UV}}
\newcommand{\bepUV}{\bar{\varepsilon}_{\rm\sss UV}}
\newcommand{\epIR}{\varepsilon_{\rm\sss IR}}
\newcommand{\OS}{{\rm OS}}
\newcommand{\BW}{{\rm BW}}
\newcommand{\stepf}{\Theta}
\newcommand\muF{\mu_{\sss F}}
\newcommand\muR{\mu_{\sss R}}
\newcommand{\ttv}{t\bar{t}V}
\newcommand{\ttV}{\ttv}
\newcommand{\nmax}{{\tt n_{\tt max}}}
\newcommand{\mmax}{{\tt m_{\tt max}}}
\newcommand\FKSpairs{{\cal P}_{\sss\rm FKS}}
\newcommand\ren{{\rm R}}
\newcommand\unren{{\rm U}}
\newcommand\oG{\overline{G}}
\newcommand\vertmuCM{{\cal V}^{(0)\mu}_{\rm CM}}
\newcommand\vertnuCM{{\cal V}^{(0)\nu}_{\rm CM}}
\newcommand\vertmuZW{{\cal V}^{(0)\mu}_{\rm ZW}}
\newcommand\vertnuZW{{\cal V}^{(0)\nu}_{\rm ZW}}
\newcommand\bM{\bar{M}}
\newcommand\bGa{\bar{\Gamma}}
\newcommand\bga{\bar{\gamma}}
\newcommand\bMW{\bM_W}
\newcommand\bGaW{\bGa_W}
\newcommand\bMZ{\bM_Z}
\newcommand\bGaZ{\bGa_Z}
\newcommand\ZW{{\rm ZW}}
\newcommand\CM{{\rm CM}}


%%%%madqed.tex
\newcommand{\bq}{\bar{q}}
\newcommand{\epem}{e^+e^-}
\newcommand{\mpmm}{\mu^+\mu^-}
\newcommand{\ord}{{\cal O}}
\newcommand{\Sfun}{{\cal S}}
\newcommand{\Sfunij}{\Sfun_{ij}}
\newcommand\asotpi{\frac{\as}{2\pi}}
\newcommand\aotpi{\frac{\aem}{2\pi}}
\newcommand\muoQ{\frac{\mu^2}{Q^2}}
\newcommand\muoQep{\left(\frac{\mu^2}{Q^2}\right)^\ep}
\newcommand\Dz{{\cal D}^{(0)}}
\newcommand\Do{{\cal D}^{(1)}}
\newcommand\madfks{{\sc\small MadFKS}}
\newcommand\Tt{{\rm T}}
\newcommand\QCD{{\rm QCD}}
\newcommand\QED{{\rm QED}}
\newcommand\proc{r}
\newcommand\nini{n_{\sss I}}
\newcommand\nlight{n_{\sss L}}
\newcommand\nlightB{\nlight^{\sss (B)}}
\newcommand\nlightR{\nlight^{\sss (R)}}
\newcommand\nlightBorR{\nlight^{\sss (B/R)}}
\newcommand\nheavy{n_{\sss H}}
\newcommand\nzero{n_\emptyset}
\newcommand\ident{{\cal I}}
\newcommand\Ione{\ident_1}
\newcommand\Itwo{\ident_2}
\newcommand\amp{{\cal A}}
\newcommand\ampmt{\amp^{(m,0)}}
\newcommand\ampnt{\amp^{(n,0)}}
\newcommand\ampnpot{\amp^{(n+1,0)}}
\newcommand\ampnl{\amp^{(n,1)}}
\newcommand\ampsq{{\cal M}}
\newcommand\ampsqmt{\ampsq^{(m,0)}}
\newcommand\ampsqnt{\ampsq^{(n,0)}}
\newcommand\ampsqnpot{\ampsq^{(n+1,0)}}
\newcommand\ampsqnl{\ampsq^{(n,1)}}
\newcommand\vampsqnl{{\cal V}^{(n,1)}}
\newcommand\hvampsqnl{\hat{\cal V}^{(n,1)}}
\newcommand\vampsqnlF{{\cal V}^{(n,1)}_{\sss FIN}}
\newcommand\hvampsqnlF{\hat{\cal V}^{(n,1)}_{\sss FIN}}
\newcommand\tampsq{\widetilde{\cal M}}
\newcommand\tampsqnt{\tampsq^{(n,0)}}
\newcommand\tampsqnpot{\tampsq^{(n+1,0)}}
\newcommand\Qop{\vec{Q}}
\newcommand\Qops{Q}
\newcommand\JetsB{J^{\nlightB}}
\newcommand\JetsR{J^{\nlightB+1}}
\newcommand\velkl{v_{kl}}
\newcommand\avg{{\cal N}}
\newcommand\xicut{\xi_{cut}}
\newcommand\deltaO{\delta_{\sss O}}
\newcommand\deltaI{\delta_{\sss I}}
\newcommand\NC{N_{\sss c}}
\newcommand\CA{C_{\sss A}}
\newcommand\CF{C_{\sss F}}
\newcommand\TF{T_{\sss F}}
\newcommand\DA{D_{\sss A}}
\newcommand\nC{n_{\sss c}}
\newcommand\NF{N_{\sss F}}
\newcommand\Nl{N_l}
\newcommand\eikint{{\cal E}}
\newcommand\phsp{d\phi}
\newcommand\phspn{\phsp_{n}}
\newcommand\phspnpo{\phsp_{n+1}}
\newcommand\mua{\mu_\alpha}

\newcommand\ampzCM{\amp^{(0)}_{\rm CM}}
\newcommand\ampzZW{\amp^{(0)}_{\rm ZW}}

%%%%fksfrag.tex
\newcommand{\bb}{\bar{b}}
\newcommand\ampsqttw{\ampsq^{(2,0)}}
\newcommand\ampsqtth{\ampsq^{(3,0)}}
\newcommand\ampsqltw{\ampsq^{(2,1)}}
\newcommand\epb{\overline{\epsilon}}
\newcommand\xii{\xi_i}
\newcommand\xij{\xi_j}
\newcommand\xik{\xi_k}
\newcommand\xia{\xi_\alpha}
\newcommand\xib{\xi_\beta}
\newcommand\xinpf{\xi_{n+4}}
\newcommand\xic{\xi_c}
\newcommand\xiip{\xi_i^\prime}
\newcommand\xicp{\xi_c^\prime}
\newcommand\cp{c^\prime}
\newcommand\yi{y_i}
\newcommand\yj{y_j}
\newcommand\ya{y_\alpha}
\newcommand\yb{y_\beta}
\newcommand\phii{\varphi_i}
\newcommand\phij{\varphi_j}
\newcommand\ynpf{y_{n+4}}
\newcommand\vol{{\cal V}}


%%%%results_section.tex
\newcommand{\pnote}[1]{ \textbf{[DP:} \textit{\color{red} #1}\textbf{]}}
\newcommand{\LOone}{\ensuremath{\LOo}\xspace}
\newcommand{\LOtwo}{\ensuremath{\LOt}\xspace}
\newcommand{\LOthree}{\ensuremath{\LOth}\xspace}
\newcommand{\LOfour}{\ensuremath{\LOf}\xspace}
\newcommand{\NLOone}{\ensuremath{\NLOo}\xspace}
\newcommand{\NLOtwo}{\ensuremath{\NLOt}\xspace}
\newcommand{\NLOthree}{\ensuremath{\NLOth}\xspace}
\newcommand{\NLOfour}{\ensuremath{\NLOf}\xspace}
\newcommand{\NLOfive}{\ensuremath{\NLOfv}\xspace}
\newcommand{\NLOgetwo}{\ensuremath{\NLOgt}\xspace}
\newcommand{\NLOgethree}{\ensuremath{\NLOgth}\xspace}

\def\NLOQCD{{\rm NLO_{QCD}}}
\def\LOQCD{{\rm LO_{QCD}}}
\def\NLOQCDt{\NLO_{{\rm QCD},t{\rm -ch.}}}
\def\NLOQCDEW{\NLO_{{\rm QCD+EW}}}
\def\NLOEW{\rm NLO_{EW}}


%%%%citation_summary.tex
\newcommand{\mh}{m_{ \sss H}}
\newcommand{\mw}{m_{ \sss W}}
\newcommand{\mz}{m_{ \sss Z}}
\newcommand{\mt}{m_{t}}
%\newcommand{\mh}{m_H}
%\newcommand{\mw}{m_W}
%newcommand{\mz}{m_Z}
%\newcommand{\mt}{m_t}
%\newcommand{\hw}{h_{\sss W}}


%%%%CMS_section.tex or whereabout
\newcommand{\SMWidth}{{\sc\small SMWidth}}
\newcommand{\FeynArts}{{\sc\small FeynArts}}
\newcommand{\HDecay}{{\sc\small HDecay}}
\newcommand{\OneLoop}{{\sc\small OneLoop}}
\newcommand{\MGaMC}{\aNLOs}
\newcommand{\MadLoop}{\ml}

%%%%%%%%%%%%%%% DENNER definitions %%%%%%%%%%%%

%% Abbreviations for environments
\def\beq{\begin{equation}}
\def\eeq{\end{equation}}
\def\beqar{\begin{eqnarray}}
\def\eeqar{\end{eqnarray}}
\def\barr#1{\begin{array}{#1}}
\def\earr{\end{array}}
\def\bfi{\begin{figure}}
\def\efi{\end{figure}}
\def\btab{\begin{table}}
\def\etab{\end{table}}
\def\bce{\begin{center}}
\def\ece{\end{center}}
\def\nn{\nonumber}
\def\nl{\nonumber\\}

%% shorthands for greek letters
\def\al{\alpha}
\def\be{\beta}
%\def\Ga{\Gamma}
\def\ga{\gamma}
\def\de{\delta}
%\def\De{\Delta}%\def\eps{\epsilon}
%\def\veps{\varepsilon}
\def\la{\lambda}
%\def\om{\omega}
%\def\Om{\Omega}
\def\si{\sigma}
\def\Si{\Sigma}
%\def\vth{\vartheta}
%\def\ieps{\ri\epsilon}

%% new commands for cross referencing
\def\refeq#1{\mbox{(\ref{#1})}}
\def\reffi#1{\mbox{Figure~\ref{#1}}}
\def\reffis#1{\mbox{Figures~\ref{#1}}}
\def\refta#1{\mbox{Table~\ref{#1}}}
\def\reftas#1{\mbox{Tables~\ref{#1}}}
\def\refse#1{\mbox{Section~\ref{#1}}}
\def\refses#1{\mbox{Sections~\ref{#1}}}
\def\refapp#1{\mbox{App.~\ref{#1}}}
\def\citere#1{\mbox{Ref.~\cite{#1}}}
\def\citeres#1{\mbox{Refs.~\cite{#1}}}

%%physical units
\newcommand{\TeV}{\unskip\,\mathrm{TeV}}
\newcommand{\GeV}{\unskip\,\mathrm{GeV}}
\newcommand{\MeV}{\unskip\,\mathrm{MeV}}
%\newcommand{\pba}{\unskip\,\mathrm{pb}}
%\newcommand{\fb}{\unskip\,\mathrm{fb}}

%% roman symbols
\newcommand{\ri}{{\mathrm{i}}}
\newcommand{\rd}{{\mathrm{d}}}
\newcommand{\rS}{{\mathrm{S}}}
\newcommand{\rR}{{\mathrm{R}}}
\newcommand{\rT}{{\mathrm{T}}}
\newcommand{\rL}{{\mathrm{L}}}

%% calligraphic symbols
\renewcommand{\O}{{\cal O}}
%\newcommand{\Oa}{\mathswitch{{\cal{O}}(\alpha)}}
%\newcommand{\Oaa}{\mathswitch{{\cal{O}}(\alpha^2)}}
\newcommand{\M}{{\cal{M}}}
\newcommand{\Mew}{{\tilde{\cal{M}}}}

% physical particles
%\def\mathswitchr#1{\relax\ifmmode{\mathrm{#1}}\else$\mathrm{#1}$\fi}
\def\mathswitchr#1{#1}
\newcommand{\PW}{\mathswitchr W}
\newcommand{\PB}{\mathswitchr B}
\newcommand{\PZ}{\mathswitchr Z}
\newcommand{\PA}{\mathswitchr A}
\newcommand{\PH}{\mathswitchr H}
\newcommand{\Pf}{\mathswitch f}
\newcommand{\Pfbar}{\mathswitch \bar f}
\newcommand{\Pe}{\mathswitchr e}
\newcommand{\Pd}{\mathswitchr d}
\newcommand{\Pu}{\mathswitchr u}
\newcommand{\Ps}{\mathswitchr s}
\newcommand{\Pc}{\mathswitchr c}
\newcommand{\Pb}{\mathswitchr b}
\newcommand{\Pt}{\mathswitchr t}
\newcommand{\Pep}{\mathswitchr {e^+}}
\newcommand{\Pem}{\mathswitchr {e^-}}
\newcommand{\PWp}{\mathswitchr {W^+}}
\newcommand{\PWm}{\mathswitchr {W^-}}
\newcommand{\PWpm}{\mathswitchr {W^\pm}}

% particle masses
\def\mathswitch#1{\relax\ifmmode#1\else$#1$\fi}
\newcommand{\MW}{\mathswitch {M_\PW}}
\newcommand{\MZ}{\mathswitch {M_\PZ}}
\newcommand{\MH}{\mathswitch {M_\PH}}
\newcommand{\MHt}{\mathswitch {M_{\PH,\Pt}}}
\newcommand{\Me}{\mathswitch {m_\Pe}}
%\newcommand{\Md}{\mathswitch {m_\Pd}}
%\newcommand{\Mu}{\mathswitch {m_\Pu}}
%\newcommand{\Ms}{\mathswitch {m_\Ps}}
%\newcommand{\Mc}{\mathswitch {m_\Pc}}
%\newcommand{\Mb}{\mathswitch {m_\Pb}}
\newcommand{\Mt}{\mathswitch {m_\Pt}}
\newcommand{\Mb}{\mathswitch {m_\Pb}}
\newcommand{\GW}{\Gamma_{\PW}}
\newcommand{\GZ}{\Gamma_{\PZ}}

\newcommand{\ntad}{n_{\rm tad}}

% shorthands for SM parameters
\newcommand{\thw}{\mathswitch {\theta_\mathrm{w}}}
\newcommand{\cw}{\mathswitch {c_\mathrm{w}}}
\newcommand{\sw}{\mathswitch {s_\mathrm{w}}}
\newcommand{\GF}{\mathswitch {G_\mu}}
\newcommand{\NCf}{\mathswitch {N_{\mathrm{C}}^f}}
\newcommand{\NCt}{\mathswitch {N_{\mathrm{C}}^{\Pt}}}

% mathematical symbols
\newcommand{\lsim}
{\mathrel{\raisebox{-.3em}{$\stackrel{\displaystyle <}{\sim}$}}}
\newcommand{\gsim}
{\mathrel{\raisebox{-.3em}{$\stackrel{\displaystyle >}{\sim}$}}}
\newcommand{\Tr}{\mathop{\mathrm{Tr}}\nolimits}
\newcommand{\SU}{\mathrm{SU}}
\newcommand{\U}{\mathrm{U}}
\newcommand{\SUtwo}{\mathrm{SU(2)}}
\newcommand{\Uone}{\mathrm{U}(1)}

% various abbreviations
\def\ie{i.e.\ }
\def\eg{e.g.\ }
\def\cf{cf.\ }

%\newcommand{\br}{{\mathrm{br}}}
%\newcommand{\QCD}{{\mathrm{QCD}}}
%\newcommand{\QED}{{\mathrm{QED}}}
%\newcommand{\SM}{{\mathrm{SM}}}
%\newcommand{\Born}{{\mathrm{Born}}}
%\newcommand{\born}{{\mathrm{Born}}}
%\newcommand{\corr}{{\mathrm{corr}}}
%\newcommand{\onel}{{\mathrm{1-loop}}}
\newcommand{\elm}{{\mathrm{em}}}
\newcommand{\ew}{{\mathrm{ew}}}
\newcommand{\sew}{{\mathrm{ew}}}
\newcommand{\htop}{{H,t}}
\newcommand{\weak}{{\mathrm{weak}}}
%\newcommand{\cut}{{\mathrm{cut}}}
%\newcommand{\SB}{{\mathrm{SB}}}
%\newcommand{\CMS}{{\mathrm{CMS}}}
%\newcommand{\CC}{{\mathrm{CC}}}
%\newcommand{\NC}{{\mathrm{NC}}}
\newcommand{\real}{{\mathrm{real}}}
\newcommand{\virt}{{\mathrm{virt}}}
\newcommand{\soft}{{\mathrm{soft}}}
\newcommand{\coll}{{\mathrm{coll}}}
\newcommand{\rem}{{\mathrm{rem}}}
\newcommand{\symm}{{\mathrm{symm}}}
\newcommand{\asymm}{{\mathrm{asymm}}}
\newcommand{\fact}{{\mathrm{fact}}}
\newcommand{\nonfact}{{\mathrm{nfact}}}
%\renewcommand{\min}{{\mathrm{min}}}
%\renewcommand{\max}{{\mathrm{max}}}
\newcommand{\SC}{{\mathrm{LSC}}}
\renewcommand{\SS}{{\mathrm{SSC}}}
\newcommand{\cc}{{\mathrm{C}}}
\newcommand{\s}{{\mathrm{s}}}
\newcommand{\pre}{{\mathrm{PR}}}
\newcommand{\Yuk}{{\mathrm{Yuk}}}

%% commands for this paper
\newcommand{\eeWW}{{\Pe^+ \Pe^-\to \PW^+ \PW^-}}
\newcommand{\Wpff}{{\PW^+ \to f_1\bar f_2}}
\newcommand{\Wmff}{{\PW^- \to f_3\bar f_4}}
\newcommand{\eeWWffff}{\Pep\Pem\to\PW\PW\to 4f}
\newcommand{\eeffff}{\Pep\Pem\to 4f}
\newcommand{\eeffffg}{\eeffff\ga}
%\newcommand{\bew}{\beta^{\ew}}
%\newcommand{\besw}{\tilde{\beta}^{\ew}}
\newcommand{\bew}{b^{\ew}}
\newcommand{\besw}{\tilde{b}^{\ew}}

\newcommand{\cew}{C^{\ew}}
\newcommand{\csew}{\tilde{C}^{\ew}}
\newcommand{\dew}{D^{\ew}}
\newcommand{\dsew}{\tilde{D}^{\ew}}
\newcommand{\sNB}{\tilde{\NB}}
\newcommand{\NB}{N}
\newcommand{\GB}{V}

% shorthands for energy dependent single logarithms
\newcommand{\ls}{l(s)}
\newcommand{\lu}{l(\mu^2)}
\newcommand{\lQ}{l(Q^2)}
\newcommand{\lrM}{l(r_{kl},M^2)}
\newcommand{\lrMwithabs}{l(|r_{kl}|,M^2)}
\newcommand{\lrW}{l(r_{kl},\MW^2)}
\newcommand{\lrZ}{l(r_{kl},\MZ^2)}
\newcommand{\lsMa}{l(s,M_{\GB_a}^2)}
\newcommand{\lsM}{l(s,M^2)}
\newcommand{\lsW}{l(s,\MW^2)}
\newcommand{\lsf}{l(s,m_f^2)}
\newcommand{\lmuf}{l(\mu^2,m_f^2)}
\newcommand{\lmuZ}{l(\mu^2,\MZ^2)}
\newcommand{\lmuW}{l(\mu^2,\MW^2)}
\newcommand{\lsl}{l_{\cc}}
\newcommand{\lpr}{l_{\pre}}
\newcommand{\lYuk}{l_{\Yuk}}
\newcommand{\lZ}{l_{\PZ}}
% shorthands for constant single logarithms
\newcommand{\lWf}{l(\MW^2,m_f^2)}
\newcommand{\lWfnew}{l^{\rm reg}(\MW^2,m_f^2)}
\newcommand{\lWfsi}{l(\MW^2,m_{f_\si}^2)}
\newcommand{\lWk}{l(\MW^2,m_k^2)}
\newcommand{\lWla}{l(\MW^2,\la^2)}
\newcommand{\lWlanew}{l(\MW^2,Q^2)}
\newcommand{\lWlanewmu}{l(\MW^2,\mu^2)}

\newcommand{\lWfsii}{l(\MW^2,m_{f_{\si,i}}^2)}
\newcommand{\lWNB}{l(\MW^2,M_\NB^2)}
\newcommand{\lWa}{l(\MW^2,M_{\GB_a}^2)}
\newcommand{\lWZ}{l(\MW^2,\MZ^2)}
\newcommand{\lWM}{l(\MW^2,M^2)}
\newcommand{\ltW}{l(\Mt^2,\MW^2)}
\newcommand{\lHW}{l(\MH^2,\MW^2)}
% shorthands for electromagnetic single logarithms
%\newcommand{\lemf}{l^\elm(\lambda^2,m_f^2)}
%\newcommand{\lemk}{l^\elm(\lambda^2,m_{k}^2)}
%\newcommand{\leme}{l^\elm(\lambda^2,m_\Pe^2)}
%\newcommand{\lemW}{l^\elm(\lambda^2,\MW^2)}
\newcommand{\lemf}{l^\elm(m_f^2)}
\newcommand{\lemftau}{l^\elm(m_{f_\tau}^2)}
\newcommand{\lemfsi}{l^\elm(m_{f_\si}^2)}
\newcommand{\lemfsinew}{l^\elm(Q^2)}
\newcommand{\lem}{l^\elm(m^2)}
\newcommand{\lemk}{l^\elm(m_{k}^2)}
\newcommand{\lemphi}{l^\elm(m_{\varphi}^2)}
\newcommand{\leme}{l^\elm(m_\Pe^2)}
\newcommand{\lemW}{l^\elm(\MW^2)}
% shorthands for energy dependent double logarithms
\newcommand{\Ls}{L(s)}
\newcommand{\LrM}{L(|r_{kl}|,M^2)}
\newcommand{\LrMnoabs}{L(r_{kl},M^2)}
\newcommand{\Lrs}{L(|r_{kl}|,s)}
\newcommand{\LrMa}{L(|r_{kl}|,M_{\GB_a}^2)}
\newcommand{\LsM}{L(s,M^2)}
\newcommand{\LsW}{L(s,\MW^2)}
\newcommand{\LsZ}{L(s,\MZ^2)}
\newcommand{\Lsla}{L(s,\lambda^2)}
% shorthands for constant double logarithms
\newcommand{\Lkla}{L(m_k^2,\la^2)}
\newcommand{\Lklanew}{L^{\rm{reg}}(m_k^2,Q^2)}

\newcommand{\LWk}{L(\MW^2,m_k^2)}
\newcommand{\LWf}{L(\MW^2,m_f^2)}
\newcommand{\LZW}{L(\MZ^2,\MW^2)}
\newcommand{\LWla}{L(\MW^2,\lambda^2)}
\newcommand{\LWlanew}{L(\MW^2,Q^2)}

% shorthands for electromagnetic double logarithms
\newcommand{\Lemk}{L^\elm(s,\lambda^2,m_k^2)}
\newcommand{\Lemknew}{L^\elm(s,Q^2,m_k^2)}

\newcommand{\Lemphi}{L^\elm(s,\lambda^2,m_\varphi^2)}
\newcommand{\Lemf}{L^\elm(s,\lambda^2,m_f^2)}
\newcommand{\Lemftau}{L^\elm(s,\lambda^2,m_{f_\tau}^2)}
\newcommand{\Leme}{L^\elm(s,\lambda^2,m_e^2)}
\newcommand{\LemW}{L^\elm(s,\lambda^2,\MW^2)}
% shorthands for angular logarithms
%\newcommand{\ltu}{\log{\left(\frac{t}{u}\right)}}
%\newcommand{\lts}{\log{\left(\frac{|t|}{s}\right)}}
%\newcommand{\lus}{\log{\left(\frac{|u|}{s}\right)}}
\newcommand{\lrs}{\log{\frac{|r_{kl}|}{s}}}
\newcommand{\lrsalpha}{l(|r_{kl}|,s)}

\newcommand{\ltu}{\log{\frac{t}{u}}}
\newcommand{\lts}{\log{\frac{|t|}{s}}}
\newcommand{\lus}{\log{\frac{|u|}{s}}}
\newcommand{\lsu}{\log{\frac{s}{|u|}}}

  \newcommand{\TO}{\rightarrow}
\newcommand{\mglong}{{\sc\small Mad\-Graph5\_aMC\-@NLO}}
\newcommand{\HELAS}{{\sc\small Helas}}
\newcommand{\ALOHA}{{\sc\small Aloha}}
\newcommand{\mgshort}{{\sc\small MG5\_aMC}}
%\newcommand{\denpoz}{Denner{\&}Pozzorini}
\newcommand{\denpoz}{{\sc\small DP}}

\newcommand{\deltaEW}{\delta^{\rm EW}_{\rm LA}}
\newcommand{\deltaQCD}{\delta^{\rm QCD}_{\rm LA}}
\newcommand{\Ltop}{L^t(s)}
\def\Las#1{l^{\as}(#1)}
\newcommand{\dmtQCD}{(\de\Mt)^{\rm QCD}}
%L^t(Q^2) + \left(\frac{n_{\as}-1}{2}-n_{g}\right) L^{\as}(Q^2)
 \newcommand{\MSbar}{{\rm \overline{MS}}}

 \usepackage{comment}


%%%%%%%%%%%%%%% DENNER definitions end %%%%%%%%%%

%%mie definizioni%%%%%
%\newcommand{\eeWW}{e^+e^-\longrightarrow W^+W^-}




\title{Updated cross sections for $t\bar t H$ and $t H$ production}


\author[x]{\ldots}
\author[c]{Marco Zaro}

\affiliation[c]{TIFLab, Universit\`a degli Studi di Milano \& INFN, Sezione di Milano, Via Celoria 16, 20133 Milano, Italy}




\emailAdd{marco.zaro@mi.infn.it}


\abstract{New reference cross sections are presented.}

%\keywords{}
%
\preprint{
\begin{flushright}
%\today
\end{flushright}
}


\SetWatermarkText{PRELIMINARY}
\SetWatermarkLightness{0.85}
\SetWatermarkScale{4.5}
\SetWatermarkAngle{55}





\begin{document}
\maketitle
\flushbottom

We report on the calculation of the updated reference cross sections for $t\bar t H$ and $t H$ production. Parameters and setups are chosen according to
\url{https://twiki.cern.ch/twiki/bin/view/LHCPhysics/LHCHWG136TeVxsec}


\section{$t\bar t H$}
Known radiative corrections for $t \bar t H$ production at hadron colliders include NLO QCD corrections~\cite{Reina:2001sf,Reina:2001bc,Dawson:2002tg},
NLO EW~\cite{Frixione:2014qaa,Zhang:2014gcy,Frixione:2015zaa} and complete NLO~\cite{Frederix:2018nkq} preictions, as well as predictions 
beyond NLO QCD, where higher-order effects are either included via soft-gluon resummation, known up to 
NNLL~\cite{Broggio:2015lya,Kulesza:2015vda,Broggio:2016lfj,Kulesza:2017ukk}, or by considering NNLO corrections with 
an approximation of the (still unknown) two-loop
amplitude~\cite{Catani:2022mfv}. We will refer to the latter as approximate-NNLO (aNNLO). Predictions obtained via resummation have been combined
with the complete-NLO prediction~\cite{ Kulesza:2018tqz,Broggio:2019ewu,Kulesza:2020nfh}. 

In the presentation of results, we will use the following labels for the different groups providing the results:
\begin{enumerate}[I:]
    \item Ref.~\cite{Broggio:2019ewu} (Broggio et al.).
        The results have been obtained using the same calculation framework of Ref.~\cite{Broggio:2019ewu} with the  recommended input parameters. 
        Predictions are therefore accurate at Complete-NLO accuracy, meaning all the QCD and EW perturbative orders stemming from tree-level 
        and one-loop diagrams, matched to soft gluon emission corrections resummed to NNLL accuracy. We stress that the 
        theory uncertainty has been obtained by combining two different dynamical-scale choices, as described in Ref.~\cite{Broggio:2019ewu}.
        In particular, these two choices for the central scale are
        \begin{equation}
            \mu_0^{(1)}=\frac{H_T}{2}\,, \qquad \mu_0^{(2)}=\frac{m(t\bar t H)}{2}\,.
        \end{equation}
        The choice $\mu_0^{(1)}$ is employed for the central value of the predictions, while the scale-uncertainty band is evaluated by considering the 
        envelope of the nine-point variations (by a factor 2 up/down) {\bf MZ CHECK!!} obtained with the two choice around the respective central value.
        
        Besides the input parameters, there is an other difference with  Ref.~\cite{Broggio:2019ewu}. Since results depend on the photon PDF, not provided
        by the recommended PDF set (PDF4LHC21\_40), the following combination has been performed: the Complete-NLO prediction computed
        with the aforementioned set has been reweighed with the ratio of the Complete-NLO prediction obtained with 
        LUXqed17\_plus\_PDF4LHC15\_nnlo\_100 over the same prediction where the photon PDF has been set to zero. The corrections from the 
        NNLL resummation have not been rescaled. We note that the impact of the photon PDF is about 0.68\% on the final result.
    \item Refs.~\cite{Kulesza:2018tqz,Kulesza:2020nfh} (Kulesza et al.).
        Scale setting: 
        \begin{equation}
            \mu_F=\mu_R=H_T/2\,.
        \end{equation}
        Scale uncertaintities: seven point method, i.e. maximal and minimal values of the seven combinations 
        \begin{equation}
        \left(\frac{\mu_F}{\mu_0},\frac{\mu_R}{\mu_0}\right)=(0.5,0.5),(0.5,1),(1,0.5),(1,1),(1,2),(2,1),(2,2)
        \end{equation}
        around the central
        value $\left(\frac{\mu_F}{\mu_0},\frac{\mu_R}{\mu_0}\right)=(1,1)$.\\
EW scheme for corrections: $G_\mu$ scheme.\\
Pdfs: PDF4LHC21\_40\_pdfas~\cite{  } for QCD-initiated channels and LUXqed17\_plus\_PDF4LHC15\_nnlo\_100~\cite{  } for the photon-initiated channels.\\

Inclusion of photon corrections: photon-initiated channels are included at NLO (output from aMC@NLO for these channels run with LUXqed17\_plus\_PDF4LHC15\_nnlo\_100 pdfs)
    \item Ref.~\cite{Catani:2022mfv} (Catani et al.). Numbers are computed at the aNNLO QCD + complete-NLO accuracy. {\bf MZ preliminary numbers
            do not include photonic contributions}.
             The QCD scale uncertainties refer to a 7-point variation around the central scale 
             \begin{equation}
                  \mu_F=\mu_R=H_T/2\,,
             \end{equation} 
             but symmetrized by understanding the maximal deviation from the central-scale prediction among the 7-point variation points 
             as the QCD scale uncertainty applied in both directions. The theoretical uncertainty contains the uncertainty due to the 
             soft-Higgs approximation that we apply for the 2-loop amplitudes as well as statistical integration uncertainties 
             and the uncertainty due to the $r_{\textrm{cut}}\to 0$ extrapolation in our $q_T$-slicing approach; it is, however, 
             always dominated by the soft-approximation uncertainty.\\
             Results for a set of different functional forms for the central scale are also presented in Tab.~\ref{tab:kallweit-scale}.
\end{enumerate}
The fixed-order part of I, II is computed using {\sc MadGraph5\_aMC@NLO}~\cite{Alwall:2014hca,Frederix:2018nkq}.

% ttH
\begin{landscape}
\begin{table}
    \centering
    \begin{tabular}{cc|ccc|ccc|ccc|ccc|cc}
         & & \multicolumn{3}{c}{$\mu=\frac{m_H}{2}+m_t$}  & 
                                            \multicolumn{3}{c}{$\mu=\sum_{t,\bar t,H}\frac{E_T}{2}$}  &    
                                            \multicolumn{3}{c}{$\mu=\frac{H_T}{2} $} &    
                                            \multicolumn{3}{c}{$\mu=\frac{m(t\bar t H)}{2} $}\\
        $\sqrt{s}$ [TeV]  &  $m_H$ [GeV]  &
        $\sigma$ [fb] & $\delta_{\mu} + \delta_{\rm ThU}$ & $\delta_{\rm PDF}$ &
        $\sigma$ [fb] & $\delta_{\mu} + \delta_{\rm ThU}$ & $\delta_{\rm PDF}$ &
        $\sigma$ [fb] & $\delta_{\mu} + \delta_{\rm ThU}$ & $\delta_{\rm PDF}$ &
        $\sigma$ [fb] & $\delta_{\mu} + \delta_{\rm ThU}$ & $\delta_{\rm PDF}$ 
        \\
        \hline
          13.0  & 124.60  &  532.0  & $^{+3.1}_{-3.1} \,\pm 0.6$ & 2.3 &   527.6  & $^{+4.3}_{-4.3} \,\pm 0.6$ & 2.3 &   503.8  & $^{+4.1}_{-4.1} \,\pm 0.7$ & 2.3 &   523.0  & $^{+4.6}_{-4.6} \,\pm 0.6$ & 2.3 & \\ 
 13.0  & 125.00  &  528.4  & $^{+3.2}_{-3.2} \,\pm 0.7$ & 2.3 &   523.8  & $^{+4.4}_{-4.4} \,\pm 0.6$ & 2.3 &   500.5  & $^{+4.2}_{-4.2} \,\pm 0.7$ & 2.3 &   519.1  & $^{+4.7}_{-4.7} \,\pm 0.6$ & 2.3 & \\ 
 13.0  & 125.09  &  526.6  & $^{+3.1}_{-3.1} \,\pm 0.7$ & 2.3 &   522.2  & $^{+4.4}_{-4.4} \,\pm 0.6$ & 2.3 &   499.0  & $^{+4.2}_{-4.2} \,\pm 0.7$ & 2.3 &   517.7  & $^{+4.6}_{-4.6} \,\pm 0.6$ & 2.3 & \\ 
 13.0  & 125.38  &  522.7  & $^{+3.1}_{-3.1} \,\pm 0.7$ & 2.3 &   518.6  & $^{+4.4}_{-4.4} \,\pm 0.7$ & 2.3 &   495.1  & $^{+4.1}_{-4.1} \,\pm 0.7$ & 2.3 &   514.1  & $^{+4.6}_{-4.6} \,\pm 0.6$ & 2.3 & \\ 
 13.0  & 125.60  &  519.9  & $^{+3.1}_{-3.1} \,\pm 0.7$ & 2.3 &   515.5  & $^{+4.3}_{-4.3} \,\pm 0.6$ & 2.3 &   492.8  & $^{+4.2}_{-4.2} \,\pm 0.7$ & 2.3 &   511.1  & $^{+4.6}_{-4.6} \,\pm 0.6$ & 2.3 & \\ 
 13.0  & 126.00  &  515.4  & $^{+3.1}_{-3.1} \,\pm 0.7$ & 2.3 &   511.2  & $^{+4.4}_{-4.4} \,\pm 0.7$ & 2.3 &   488.8  & $^{+4.2}_{-4.2} \,\pm 0.7$ & 2.3 &   506.7  & $^{+4.6}_{-4.6} \,\pm 0.6$ & 2.3 & \\ 
 13.6  & 124.60  &  596.6  & $^{+3.0}_{-3.0} \,\pm 0.7$ & 2.2 &   592.1  & $^{+4.3}_{-4.3} \,\pm 0.6$ & 2.2 &   565.6  & $^{+4.1}_{-4.1} \,\pm 0.6$ & 2.2 &   587.0  & $^{+4.6}_{-4.6} \,\pm 0.6$ & 2.2 & \\ 
 13.6  & 125.00  &  589.9  & $^{+2.9}_{-2.9} \,\pm 0.7$ & 2.2 &   585.9  & $^{+4.3}_{-4.3} \,\pm 0.6$ & 2.2 &   559.0  & $^{+4.0}_{-4.0} \,\pm 0.6$ & 2.2 &   580.9  & $^{+4.5}_{-4.5} \,\pm 0.6$ & 2.2 & \\ 
 13.6  & 125.09  &  589.6  & $^{+3.0}_{-3.0} \,\pm 0.7$ & 2.2 &   585.3  & $^{+4.3}_{-4.3} \,\pm 0.6$ & 2.2 &   559.0  & $^{+4.1}_{-4.1} \,\pm 0.6$ & 2.2 &   580.3  & $^{+4.6}_{-4.6} \,\pm 0.6$ & 2.2 & \\ 
 13.6  & 125.38  &  586.2  & $^{+3.0}_{-3.0} \,\pm 0.7$ & 2.2 &   581.6  & $^{+4.3}_{-4.3} \,\pm 0.6$ & 2.2 &   555.4  & $^{+4.1}_{-4.1} \,\pm 0.6$ & 2.2 &   576.6  & $^{+4.6}_{-4.6} \,\pm 0.6$ & 2.2 & \\ 
 13.6  & 125.60  &  583.5  & $^{+3.0}_{-3.0} \,\pm 0.7$ & 2.2 &   579.1  & $^{+4.3}_{-4.3} \,\pm 0.7$ & 2.2 &   553.2  & $^{+4.1}_{-4.1} \,\pm 0.7$ & 2.2 &   574.0  & $^{+4.6}_{-4.6} \,\pm 0.6$ & 2.2 & \\ 
 13.6  & 126.00  &  577.9  & $^{+3.1}_{-3.1} \,\pm 0.7$ & 2.2 &   573.2  & $^{+4.3}_{-4.3} \,\pm 0.6$ & 2.2 &   547.2  & $^{+4.1}_{-4.1} \,\pm 0.7$ & 2.2 &   568.1  & $^{+4.6}_{-4.6} \,\pm 0.6$ & 2.2 & \\ 
 14.0  & 124.60  &  639.7  & $^{+2.9}_{-2.9} \,\pm 0.7$ & 2.2 &   634.9  & $^{+4.2}_{-4.2} \,\pm 0.6$ & 2.2 &   606.2  & $^{+4.0}_{-4.0} \,\pm 0.6$ & 2.2 &   629.6  & $^{+4.5}_{-4.5} \,\pm 0.6$ & 2.2 & \\ 
 14.0  & 125.00  &  636.1  & $^{+3.0}_{-3.0} \,\pm 0.6$ & 2.2 &   631.1  & $^{+4.3}_{-4.3} \,\pm 0.6$ & 2.2 &   602.3  & $^{+4.1}_{-4.1} \,\pm 0.6$ & 2.2 &   625.7  & $^{+4.5}_{-4.5} \,\pm 0.6$ & 2.2 & \\ 
 14.0  & 125.09  &  633.3  & $^{+2.9}_{-2.9} \,\pm 0.6$ & 2.2 &   628.5  & $^{+4.2}_{-4.2} \,\pm 0.6$ & 2.2 &   600.1  & $^{+4.0}_{-4.0} \,\pm 0.6$ & 2.2 &   623.1  & $^{+4.5}_{-4.5} \,\pm 0.6$ & 2.2 & \\ 
 14.0  & 125.38  &  632.4  & $^{+3.1}_{-3.1} \,\pm 0.6$ & 2.2 &   627.0  & $^{+4.3}_{-4.3} \,\pm 0.6$ & 2.2 &   598.6  & $^{+4.1}_{-4.1} \,\pm 0.6$ & 2.2 &   621.5  & $^{+4.6}_{-4.6} \,\pm 0.6$ & 2.2 & \\ 
 14.0  & 125.60  &  627.9  & $^{+3.0}_{-3.0} \,\pm 0.6$ & 2.2 &   622.5  & $^{+4.3}_{-4.3} \,\pm 0.6$ & 2.2 &   594.2  & $^{+4.1}_{-4.1} \,\pm 0.6$ & 2.2 &   617.1  & $^{+4.5}_{-4.5} \,\pm 0.6$ & 2.2 & \\ 
 14.0  & 126.00  &  621.2  & $^{+3.0}_{-3.0} \,\pm 0.7$ & 2.2 &   616.5  & $^{+4.2}_{-4.2} \,\pm 0.6$ & 2.2 &   588.6  & $^{+4.1}_{-4.1} \,\pm 0.6$ & 2.2 &   611.1  & $^{+4.5}_{-4.5} \,\pm 0.6$ & 2.2 & \\ 
 
    \end{tabular}
    \caption{\label{tab:tth} Predictions for the process $t \bar t H$ from group III, with different functional form for the central scale.}
\end{table}
\end{landscape}

% ttH-table kallweit scales
\begin{table}
    \centering
    \begin{tabular}{cccccccccccccc}
        $\sqrt{s}$ [TeV]  &  $m_H$ [GeV]  & \multicolumn{3}{c}{$\sigma$ [fb]}  & \multicolumn{3}{c}{$\delta_{\mu}$ ($+\delta_{\rm ThU}$)}   &  \multicolumn{3}{c}{$\delta_{\rm PDF}$}   
        & \multicolumn{3}{c}{$\delta_{\alpha_s}$}\\
         & & I & II & III & I & II & III & I & II & III & I & II & III\\
        \hline
          13.0  & 124.60  &  504.7  &  509.5  &  503.8  & $^{+7.8}_{-5.9}$ & $^{+5.4}_{-6.1}$ & $^{+4.1}_{-4.1} \,\pm 0.7$ & 2.3  & 2.2  & 2.3  & 0.0  & 1.5  & 1.6    \\ 
 13.0  & 125.00  &  499.9  &  505.1  &  500.5  & $^{+7.8}_{-5.9}$ & $^{+5.4}_{-6.1}$ & $^{+4.2}_{-4.2} \,\pm 0.7$ & 2.3  & 2.2  & 2.3  & 0.0  & 1.5  & 1.6    \\ 
 13.0  & 125.09  &  499.0  &  503.1  &  499.0  & $^{+7.9}_{-5.9}$ & $^{+5.4}_{-6.0}$ & $^{+4.2}_{-4.2} \,\pm 0.7$ & 2.3  & 2.2  & 2.3  & 0.0  & 1.5  & 1.6    \\ 
 13.0  & 125.38  &  495.5  &  500.6  &  495.1  & $^{+7.9}_{-5.9}$ & $^{+5.4}_{-6.1}$ & $^{+4.1}_{-4.1} \,\pm 0.7$ & 2.3  & 2.2  & 2.3  & 0.0  & 1.5  & 1.6    \\ 
 13.0  & 125.60  &  493.4  &  497.9  &  492.8  & $^{+7.8}_{-5.9}$ & $^{+5.4}_{-6.1}$ & $^{+4.2}_{-4.2} \,\pm 0.7$ & 2.3  & 2.2  & 2.3  & 0.0  & 1.5  & 1.6    \\ 
 13.0  & 126.00  &  488.7  &  493.7  &  488.8  & $^{+7.8}_{-5.9}$ & $^{+5.3}_{-6.1}$ & $^{+4.2}_{-4.2} \,\pm 0.7$ & 2.3  & 2.2  & 2.3  & 0.0  & 1.5  & 1.6    \\ 
 13.6  & 124.60  &  565.4  &  570.9  &  565.6  & $^{+7.8}_{-5.9}$ & $^{+5.5}_{-6.1}$ & $^{+4.1}_{-4.1} \,\pm 0.6$ & 2.2  & 2.1  & 2.2  & 0.0  & 1.5  & 1.6    \\ 
 13.6  & 125.00  &  560.4  &  566.6  &  559.0  & $^{+7.9}_{-5.9}$ & $^{+5.4}_{-6.1}$ & $^{+4.0}_{-4.0} \,\pm 0.6$ & 2.2  & 2.1  & 2.2  & 0.0  & 1.5  & 1.6    \\ 
 13.6  & 125.09  &  559.4  &  564.3  &  559.0  & $^{+7.9}_{-5.9}$ & $^{+5.5}_{-6.1}$ & $^{+4.1}_{-4.1} \,\pm 0.6$ & 2.2  & 2.1  & 2.2  & 0.0  & 1.5  & 1.6    \\ 
 13.6  & 125.38  &  556.0  &  561.5  &  555.4  & $^{+7.9}_{-5.9}$ & $^{+5.4}_{-6.1}$ & $^{+4.1}_{-4.1} \,\pm 0.6$ & 2.2  & 2.1  & 2.2  & 0.0  & 1.5  & 1.6    \\ 
 13.6  & 125.60  &  552.8  &  558.7  &  553.2  & $^{+8.1}_{-5.9}$ & $^{+5.4}_{-6.1}$ & $^{+4.1}_{-4.1} \,\pm 0.7$ & 2.2  & 2.1  & 2.2  & 0.0  & 1.5  & 1.6    \\ 
 13.6  & 126.00  &  548.2  &  553.2  &  547.2  & $^{+7.9}_{-6.0}$ & $^{+5.4}_{-6.1}$ & $^{+4.1}_{-4.1} \,\pm 0.7$ & 2.2  & 2.1  & 2.2  & 0.0  & 1.5  & 1.6    \\ 
 14.0  & 124.60  &  608.6  &  614.7  &  606.2  & $^{+8.0}_{-5.9}$ & $^{+5.5}_{-6.2}$ & $^{+4.0}_{-4.0} \,\pm 0.6$ & 2.2  & 2.1  & 2.2  & 0.0  & 1.5  & 1.6    \\ 
 14.0  & 125.00  &  603.2  &  609.2  &  602.3  & $^{+8.0}_{-5.9}$ & $^{+5.4}_{-6.1}$ & $^{+4.1}_{-4.1} \,\pm 0.6$ & 2.2  & 2.1  & 2.2  & 0.0  & 1.5  & 1.6    \\ 
 14.0  & 125.09  &  601.7  &  607.7  &  600.1  & $^{+7.9}_{-6.0}$ & $^{+5.5}_{-6.2}$ & $^{+4.0}_{-4.0} \,\pm 0.6$ & 2.2  & 2.1  & 2.2  & 0.0  & 1.5  & 1.6    \\ 
 14.0  & 125.38  &  598.4  &  603.0  &  598.6  & $^{+7.9}_{-6.0}$ & $^{+5.5}_{-6.1}$ & $^{+4.1}_{-4.1} \,\pm 0.6$ & 2.2  & 2.1  & 2.2  & 0.0  & 1.5  & 1.6    \\ 
 14.0  & 125.60  &  595.6  &  600.0  &  594.2  & $^{+7.9}_{-6.0}$ & $^{+5.5}_{-6.1}$ & $^{+4.1}_{-4.1} \,\pm 0.6$ & 2.2  & 2.1  & 2.2  & 0.0  & 1.5  & 1.6    \\ 
 14.0  & 126.00  &  589.5  &  595.2  &  588.6  & $^{+7.9}_{-6.0}$ & $^{+5.4}_{-6.1}$ & $^{+4.1}_{-4.1} \,\pm 0.6$ & 2.2  & 2.1  & 2.2  & 0.0  & 1.5  & 1.6    \\ 
 
    \end{tabular}
    \caption{\label{tab:tth} Predictions for the process $t \bar t H$.}
\end{table}
\clearpage


\section{$tH$}
As far as $tH$ is concerned, three main production mechanisms concur: $t$-channel, $s$-channel and $tWH$ associated production. State-of-the-art predictions
for these processes include corrections up to NLO QCD~\cite{Demartin:2015uha,Demartin:2016axk} as well as NLO EW~\cite{Pagani:2020mov}. When
the latter are considered, however, interferences between the three production channels cannot be neglected. Thus, the impact of EW corrections can be only assessed
by considering all the channels together (possibly imposing the selection cuts which enhance a given channel). In this case, the effect on the inclusive
cross section has been found in Ref.~\cite{Pagani:2020mov} to be about $-3.5\%$. Given the rather low rate of these processes, EW
corrections are not included in the reference cross sections.\\

All single-top and Higgs cross-sections presented below are computed with {\sc MadGraph5\_aMC@NLO}~\cite{Alwall:2014hca,Frederix:2018nkq}. 

\subsection{$t$-channel}
We show predictions for $t$-channel single-top and Higgs associated production in Tabs.~\ref{tab:th-t-tot}, \ref{tab:th-t-top}~and~\ref{tab:th-t-atop},
respectively for the processes $tH + \bar t H$, $tH$ and $\bar t H$, at NLO QCD accuracy, following Ref.~\cite{Demartin:2015uha}. The 
central value of the cross section is computed in the five-flavour scheme. 
The central value of the renormalisation and factorisation scales is set to
\begin{equation}
    \mu_{R,F}^0 = \frac{m_H+m_t}{4}\,.
\end{equation}
The flavour-scheme and scale-variation uncertainty is computed by considering the envelope of the 
nine-point scale variations (obtained varying $ \mu_{R,F}^0$ by a factor 2 up and down) in the four- and five-flavour schemes.

% single top, t-channel
\begin{table}
    \centering
    \begin{tabular}{cccccc}
        $\sqrt{s}$ [TeV]  &  $m_H$ [GeV]  &  $\sigma$ [fb]  & $\delta_{\mu}$   &  $\delta_{\rm PDF}$   & $\delta_{\alpha_s}$\\
        \hline
          13.0  & 125.00  &  76.04  & $^{+6.4}_{-15.9}$ & 1.8  & 1.2  \\ 
 13.6  & 125.00  &  85.38  & $^{+6.4}_{-15.5}$ & 1.7  & 1.2  \\ 
 14.0  & 125.00  &  92.02  & $^{+6.3}_{-15.0}$ & 1.7  & 1.2  \\ 
 
    \end{tabular}
    \caption{\label{tab:th-t-tot} Predictions for the process $tH + \bar t H$, $t$-channel.}
\end{table}
%
\begin{table}
    \centering
    \begin{tabular}{cccccc}
        $\sqrt{s}$ [TeV]  &  $m_H$ [GeV]  &  $\sigma$ [fb]  & $\delta_{\mu}$   &  $\delta_{\rm PDF}$   & $\delta_{\alpha_s}$\\
        \hline
          13.0  & 125.00  &  49.91  & $^{+6.5}_{-15.2}$ & 1.5  & 1.2  \\ 
 13.6  & 125.00  &  56.00  & $^{+6.4}_{-15.2}$ & 1.5  & 1.2  \\ 
 14.0  & 125.00  &  60.14  & $^{+6.3}_{-14.6}$ & 1.4  & 1.1  \\ 
 
    \end{tabular}
    \caption{\label{tab:th-t-top} Predictions for the process $tH$, $t$-channel.}
\end{table}
%
\begin{table}
    \centering
    \begin{tabular}{cccccc}
        $\sqrt{s}$ [TeV]  &  $m_H$ [GeV]  &  $\sigma$ [fb]  & $\delta_{\mu}$   &  $\delta_{\rm PDF}$   & $\delta_{\alpha_s}$\\
        \hline
          13.0  & 125.00  &  26.13  & $^{+6.4}_{-17.1}$ & 3.1  & 1.3  \\ 
 13.6  & 125.00  &  29.38  & $^{+6.3}_{-16.3}$ & 3.0  & 1.2  \\ 
 14.0  & 125.00  &  31.88  & $^{+6.3}_{-15.8}$ & 2.9  & 1.2  \\ 
 
    \end{tabular}
    \caption{\label{tab:th-t-atop} Predictions for the process $\bar tH$, $t$-channel.}
\end{table}


\subsection{$s$-channel}
We show predictions for $s$-channel single-top and Higgs associated production in Tabs.~\ref{tab:th-s-tot}, \ref{tab:th-s-top}~and~\ref{tab:th-s-atop},
respectively for the processes $tH + \bar t H$, $tH$ and $\bar t H$, at NLO QCD accuracy, following Ref.~\cite{Demartin:2015uha}. The central 
value of the cross section is computed in the five-flavour scheme. 
The central value of the renormalisation and factorisation scales is set to
\begin{equation}
    \mu_{R,F}^0 = \frac{m_H+m_t}{2}\,.
\end{equation}
The scale-variation uncertainty is computed by considering the envelope of the 
nine-point scale variations (obtained varying $ \mu_{R,F}^0$ by a factor 2 up and down).

% single top, s-channel
\begin{table}
    \centering
    \begin{tabular}{cccccc}
        $\sqrt{s}$ [TeV]  &  $m_H$ [GeV]  &  $\sigma$ [fb]  & $\delta_{\mu}$   &  $\delta_{\rm PDF}$   & $\delta_{\alpha_s}$\\
        \hline
          13.0  & 125.00  &   2.93  & $^{+2.4}_{-1.9}$ & 2.4  & 0.2  \\ 
 13.6  & 125.00  &   3.15  & $^{+2.4}_{-1.8}$ & 2.3  & 0.2  \\ 
 14.0  & 125.00  &   3.30  & $^{+2.4}_{-1.8}$ & 2.3  & 0.3  \\ 
 
    \end{tabular}
    \caption{\label{tab:th-s-tot} Predictions for the process $tH + \bar t H$, $s$-channel.}
\end{table}
%
\begin{table}
    \centering
    \begin{tabular}{cccccc}
        $\sqrt{s}$ [TeV]  &  $m_H$ [GeV]  &  $\sigma$ [fb]  & $\delta_{\mu}$   &  $\delta_{\rm PDF}$   & $\delta_{\alpha_s}$\\
        \hline
          13.0  & 124.60  &   1.93  & $^{+2.4}_{-1.8}$ & 2.5  & 0.2  \\ 
 13.0  & 125.00  &   1.92  & $^{+2.4}_{-1.8}$ & 2.5  & 0.2  \\ 
 13.0  & 125.09  &   1.91  & $^{+2.4}_{-1.8}$ & 2.5  & 0.2  \\ 
 13.0  & 125.38  &   1.90  & $^{+2.4}_{-1.8}$ & 2.5  & 0.2  \\ 
 13.0  & 125.60  &   1.90  & $^{+2.4}_{-1.8}$ & 2.5  & 0.2  \\ 
 13.0  & 126.00  &   1.88  & $^{+2.4}_{-1.8}$ & 2.5  & 0.2  \\ 
 13.6  & 124.60  &   2.07  & $^{+2.4}_{-1.8}$ & 2.4  & 0.3  \\ 
 13.6  & 125.00  &   2.06  & $^{+2.4}_{-1.8}$ & 2.5  & 0.3  \\ 
 13.6  & 125.09  &   2.05  & $^{+2.4}_{-1.8}$ & 2.5  & 0.3  \\ 
 13.6  & 125.38  &   2.05  & $^{+2.4}_{-1.8}$ & 2.5  & 0.3  \\ 
 13.6  & 125.60  &   2.04  & $^{+2.4}_{-1.8}$ & 2.5  & 0.3  \\ 
 13.6  & 126.00  &   2.03  & $^{+2.4}_{-1.8}$ & 2.5  & 0.3  \\ 
 14.0  & 124.60  &   2.17  & $^{+2.3}_{-1.7}$ & 2.4  & 0.3  \\ 
 14.0  & 125.00  &   2.15  & $^{+2.3}_{-1.8}$ & 2.4  & 0.3  \\ 
 14.0  & 125.09  &   2.15  & $^{+2.4}_{-1.8}$ & 2.4  & 0.3  \\ 
 14.0  & 125.38  &   2.14  & $^{+2.4}_{-1.8}$ & 2.4  & 0.3  \\ 
 14.0  & 125.60  &   2.13  & $^{+2.3}_{-1.8}$ & 2.4  & 0.3  \\ 
 14.0  & 126.00  &   2.11  & $^{+2.3}_{-1.7}$ & 2.4  & 0.3  \\ 
 
    \end{tabular}
    \caption{\label{tab:th-s-top} Predictions for the process $tH$, $s$-channel.}
\end{table}
%
\begin{table}
    \centering
    \begin{tabular}{cccccc}
        $\sqrt{s}$ [TeV]  &  $m_H$ [GeV]  &  $\sigma$ [fb]  & $\delta_{\mu}$   &  $\delta_{\rm PDF}$   & $\delta_{\alpha_s}$\\
        \hline
          13.0  & 124.60  &   1.02  & $^{+2.5}_{-1.9}$ & 2.6  & 0.2  \\ 
 13.0  & 125.00  &   1.01  & $^{+2.5}_{-1.9}$ & 2.6  & 0.2  \\ 
 13.0  & 125.09  &   1.00  & $^{+2.5}_{-1.8}$ & 2.6  & 0.2  \\ 
 13.0  & 125.38  &   1.00  & $^{+2.5}_{-1.9}$ & 2.6  & 0.2  \\ 
 13.0  & 125.60  &   1.00  & $^{+2.4}_{-1.8}$ & 2.6  & 0.2  \\ 
 13.0  & 126.00  &   0.99  & $^{+2.5}_{-1.9}$ & 2.6  & 0.2  \\ 
 13.6  & 124.60  &   1.10  & $^{+2.4}_{-1.8}$ & 2.5  & 0.2  \\ 
 13.6  & 125.00  &   1.09  & $^{+2.4}_{-1.8}$ & 2.5  & 0.2  \\ 
 13.6  & 125.09  &   1.09  & $^{+2.4}_{-1.8}$ & 2.5  & 0.2  \\ 
 13.6  & 125.38  &   1.09  & $^{+2.4}_{-1.8}$ & 2.5  & 0.2  \\ 
 13.6  & 125.60  &   1.08  & $^{+2.4}_{-1.8}$ & 2.5  & 0.2  \\ 
 13.6  & 126.00  &   1.07  & $^{+2.4}_{-1.8}$ & 2.5  & 0.2  \\ 
 14.0  & 124.60  &   1.16  & $^{+2.5}_{-1.8}$ & 2.5  & 0.3  \\ 
 14.0  & 125.00  &   1.15  & $^{+2.4}_{-1.8}$ & 2.5  & 0.2  \\ 
 14.0  & 125.09  &   1.15  & $^{+2.4}_{-1.8}$ & 2.5  & 0.3  \\ 
 14.0  & 125.38  &   1.14  & $^{+2.4}_{-1.8}$ & 2.5  & 0.3  \\ 
 14.0  & 125.60  &   1.14  & $^{+2.4}_{-1.8}$ & 2.5  & 0.2  \\ 
 14.0  & 126.00  &   1.13  & $^{+2.5}_{-1.8}$ & 2.5  & 0.2  \\ 
 
    \end{tabular}
    \caption{\label{tab:th-s-atop} Predictions for the process $\bar tH$, $s$-channel.}
\end{table}

\subsection{$tWH$}
We show predictions $tWH$ associated production in Tabs.~\ref{tab:thw-tot}, where the cross section for the process $tHW^-$ and $\bar t HW^+$ are summed. Each
process, taken alone, contributes to half the cross section. Numbers have been computed at NLO QCD accuracy following Ref.~\cite{Demartin:2016axk}. In particular, the ``diagram removal with interference'' (DR2)
approach, as implemented in the {\sc MadSTR} plugin~\cite{Frixione:2019fxg}, is employed to subtract contributions from resonant top quarks which appear at NLO.
The central value of the renormalisation and factorisation scales is set to
\begin{equation}
    \mu_{R,F}^0 = \frac{m_H+m_t+m_W}{2}\,.
\end{equation}
The scale-variation uncertainty is computed by considering the envelope of the 
nine-point scale variations (obtained varying $ \mu_{R,F}^0$ by a factor 2 up and down).

\begin{table}
    \centering
    \begin{tabular}{cccccc}
        $\sqrt{s}$ [TeV]  &  $m_H$ [GeV]  &  $\sigma$ [fb]  & $\delta_{\mu}$   &  $\delta_{\rm PDF}$   & $\delta_{\alpha_s}$\\
        \hline
         \input{table_tH-w__tot.tex} 
    \end{tabular}
    \caption{\label{tab:thw-tot} Predictions for the process $tHW^- + \bar t HW^+$ (with DR2). The rate of each of the two processes taken alone is
half of the rate of their sum.}
\end{table}


\addcontentsline{toc}{section}{References}
\bibliographystyle{JHEP}
\bibliography{biblio}

\end{document}

